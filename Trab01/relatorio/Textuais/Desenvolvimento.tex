\chapter{Desenvolvimento}

\section{Definição do Produto e Fluido Refrigerante}

Para o desenvolvimento do projeto, definiu-se como produto a ser refrigerado o pescado fresco, recebido na câmara frigorífica a 0°C e armazenado em temperatura de conservação de -25°C, condição típica para armazenamento comercial de longo prazo. As propriedades termofísicas do produto, incluindo calor específico, densidade e calor latente de congelamento, bem como as temperaturas de operação, encontram-se especificadas na Tabela~\ref{massa peixe}.

A seleção do fluido refrigerante considerou critérios técnicos, econômicos e ambientais. O refrigerante R-134a (1,1,1,2-tetrafluoretano) foi escolhido por apresentar características adequadas à faixa de temperatura requerida, baixo custo relativo, ampla disponibilidade no mercado e extensa utilização em aplicações de refrigeração comercial. Adicionalmente, o R-134a possui potencial de destruição da camada de ozônio (ODP) nulo, embora apresente potencial de aquecimento global (GWP) moderado de 1430, sendo considerado um fluido de transição aceitável para esta aplicação.

As propriedades termodinâmicas do R-134a, incluindo entalpia, entropia, temperatura e pressão em diferentes estados do ciclo, foram determinadas através da biblioteca CoolProp, uma base de dados de código aberto amplamente validada. A integração com rotinas computacionais desenvolvidas em Python permitiu o cálculo automatizado dos ciclos de refrigeração e a análise paramétrica do sistema.

\section{Estimativa da taxa de calor necessário de resfriamento}

Para a escolha do compressor a ser utilizado, primeiramente foi estimado a taxa de calor a ser retirada do sistema, a partir da Equação \ref{Q resfriamento}. Como temos o volume, o tempo de pulldown e o material a ser refigerado, podemos calcular a carga térmica mínima necessária.

\begin{equation}
    m_{peixe} = \rho V_{refrigerador}
    \label{massa peixe}
\end{equation}

\begin{equation}
    \dot{Q} = m c \Delta T / \Delta t
    \label{Q resfriamento}
\end{equation}


\begin{table}[ht]
\centering
\begin{tabular}{|c|c|}
\hline
$\rho$ {[}kg/m³{]} & 972 \\ \hline
$V_{refrigerador}$ {[}m³{]}     & 0.14 \\ \hline
$c$ {[}J/kgK{]}     &  1.71 \\ \hline
$\Delta T$ {[}K{]}     &  25 \\ \hline
$\Delta t$ {[}s{]}     &  $2.88 \cdot10^{4}$ \\ \hline
\end{tabular}
\caption{Valores utilizados.}
\label{tab:tabela dados}
\end{table}

Obtemos:
\begin{equation}
    \dot{Q} = 202.6 W
    \label{carga}
\end{equation}

Com a carga térmica definida, é necessário selecionar um compressor adequado para a operação. Para isso, será utilizado o seletor de produtos disponível no site do fabricante Embraco \textcopyright. Para a aplicação em questão, que envolve baixas temperaturas, recomenda-se a utilização de compressores do tipo LBP. Uma vez selecionados os compressores que atendiam aos requisitos, os dados de operação individuais foram obtidos no site do fabricante.

\begin{figure}
    \centering
    \includegraphics[width=0.8\linewidth]{Imagens/Desenvolvimento/PSS-embraco.png}
    \caption{Seletor de produtos.}
    \label{fig:seletor de produtos}
\end{figure}

\newpage

O documento contém as temperaturas de condensação e evaporação empregadas nos testes de desempenho, além de parâmetros como capacidade de refrigeração, consumo de energia, corrente elétrica, entre outros. Dessa maneina, foram selecionas os compressores descrtitos na Tabela \ref{tab:compressores escolhidos}.


\begin{table}[ht]
\centering
\begin{tabular}{|c|c|c|}
\hline
Modelo    & Potencia {[}W{]} & Custo {[}R\$\} \\ \hline
EGAS80HLR & 240              & 650            \\ \hline
EGZS60HLP & 180              & 1340           \\ \hline
EGZS70HLC & 202              & 1130           \\ \hline
FFU70HAK  & 221              & 600            \\ \hline
\end{tabular}
\caption{Compressores escolhidos.}
\label{tab:compressores escolhidos}
\end{table}

\section{Ciclo de Refrigeração Padrão:}

Com os dados preliminares obtidos, foi desenvolvida uma rotina em Python para calcular as propriedades do sistema, de acordo com o ciclo descrito na Figura \ref{fig:ciclo padrão}. 

\begin{figure}
    \centering
    \includegraphics[width=0.6\linewidth]{Imagens/Desenvolvimento/Diagrama.png}
    \caption{Ciclo padrão.}
    \label{fig:ciclo padrão}
\end{figure}

\newpage

O programa utiliza um método iterativo para determinar a  mínima temperatura operacional possível, com o objetivo de reduzir custos ao evitar o superdimensionamento do compressor, utilizando como base os parâmetros descritos a seguir:

\begin{equation}
    \dot{Q_L} = \dot{m}(h_1-h_4)
    \label{QL}
\end{equation}

\begin{equation}
    \dot{Q_H} = \dot{m}(h_2-h_3)
    \label{QH}
\end{equation}

    Onde $\dot{Q_L}$ e $\dot{Q_H}$, são  a taxa com que sai e que entra calor no sistema, respectivamente. O trabalho do compressor pode ser calculado apartir da equação \ref{W compressor}, utilizando as propriedades do fluido antes e depois da compresão.

\begin{equation}
    \dot{W_{comp}} = \dot{m}(h_2-h_1)
    \label{W compressor}
\end{equation}

\begin{equation}
    \Delta h_{1-2} \simeq  c_p (T_2-T_1)
    \label{simplificacao entalpia}
\end{equation}

    E a eficiência do sistema é dada como:

\begin{equation}
    COP = \frac{T_H}{T_H - T_L}
    \label{COP carnot}
\end{equation}

\newpage    

Simultaneamente, são calculadas as propriedades do fluido refrigerante em cada estado do ciclo de refrigeração. Para efeito de comparação com o ciclo real, também é realizado o cálculo do ciclo ideal de Carnot, a fim de se obter a eficiência máxima possível e as temperaturas mínimas requeridas pelo sistema.

Depois de uma rodada de anállises, dois compressores destacaram-se, o desempenho de ambos nos ciclos pode ser visto nas figuras \ref{fig:ciclo comp 1} e \ref{fig:ciclo comp 2}.

\begin{figure}[ht]
    \centering
    \includegraphics[width=0.9\linewidth]{Imagens/Desenvolvimento/ciclo_EGZS70HLC_202W.png}
    \caption{Ciclo para o EGZS70HLC.}
    \label{fig:ciclo comp 1}
\end{figure}

\begin{figure}[ht]
    \centering
    \includegraphics[width=0.9\linewidth]{Imagens/Desenvolvimento/ciclo_FFU70HAK_221W.png}
    \caption{Ciclo para o FFU70HAK.}
    \label{fig:ciclo comp 2}
\end{figure}

É possivél notar uma grande diferença entre o ciclo real e o ótimo, causada pelas perdar do sistema real, que aparecem em forma de calor. A rotina desenvolvida também calcula outros parâmetros de desempenho do sistema, tais como fluxo mássico, potência, COP e ${Q_L}$.

\newpage

\begin{figure}[ht]
    \centering
    \includegraphics[width=0.8\linewidth]{Imagens/Desenvolvimento/barras_m.png}
    \caption{Compração do $\dot{m}$.}
    \label{fig:barras fluxo massa}
\end{figure}

\begin{figure}[ht]
    \centering
    \includegraphics[width=0.8\linewidth]{Imagens/Desenvolvimento/barras_W.png}
    \caption{Compração da potência.}
    \label{fig:barras W}
\end{figure}

\newpage

\begin{figure}[ht]
    \centering
    \includegraphics[width=0.8\linewidth]{Imagens/Desenvolvimento/barras_COP.png}
    \caption{Compração do COP.}
    \label{fig:barras COP}
\end{figure}

\begin{figure}[ht]
    \centering
    \includegraphics[width=0.8\linewidth]{Imagens/Desenvolvimento/barras_QL.png}
    \caption{Compração do $Q_L$.}
    \label{fig:barras Ql}
\end{figure}



Os resultados obtidos e mostrados nas Figuras \ref{fig:barras fluxo massa} a \ref{fig:barras Ql} demonstram a fidelidade do modelo computacional calculado com a base teórica, com  cada propriedade apresentando um comportamento esperado em cada situação.

    \begin{itemize}
        \item $\dot{m}$ : O ciclo ideal apresentou o menor valor para o fluxo mássico, enquanto os ciclos reais e ótimos aparecem com valores ligeiramente maiores, isso acontece pela necessidade de uma maior retirada de calor no sistema.
        \item COP : A máxima eficiência possível é determinada pelo ciclo de Carnot de refrigeração. A discrepância entre esse valor teórico e o desempenho no sistema real indica o quanto ele se afasta do ideal.
    \end{itemize}

