
\chapter{Introdução}

A refrigeração industrial desempenha papel fundamental na cadeia de alimentos, garantindo a preservação da qualidade e segurança de produtos perecíveis durante o armazenamento e transporte. No setor pesqueiro, câmaras frigoríficas são essenciais para manter as características organolépticas do pescado, retardando processos de deterioração microbiológica e enzimática através do controle rigoroso de temperatura.

O dimensionamento adequado de sistemas de refrigeração para câmaras frigoríficas requer a determinação precisa da carga térmica total e a seleção criteriosa do compressor, componente responsável pela maior parcela do consumo energético do sistema. A escolha inadequada pode resultar em superdimensionamento, com custos de investimento e operação elevados, ou subdimensionamento, comprometendo a capacidade de refrigeração e a qualidade do produto armazenado.

Este projeto visa ao dimensionamento de um sistema de refrigeração por compressão de vapor para câmaras frigoríficas destinadas ao armazenamento de pescado. A partir das especificações de temperatura de armazenamento, volume interno da câmara e tempo de resfriamento requerido, será determinada a carga térmica total do sistema com o objetivo de selecionar o compressor com melhor desempenho termodinâmico e eficiência energética para a aplicação definida.

A metodologia adotada inicia-se com o levantamento das propriedades termofísicas do pescado, incluindo calor específico e densidade, permitindo quantificar os requisitos de resfriamento do produto. Em seguida, foi elaborado um banco de dados com compressores candidatos utilizando informações técnicas disponibilizadas pelo Product Selector da Embraco. A análise comparativa foi realizada em três etapas progressivas: primeiramente, considerou-se um ciclo de refrigeração ideal baseado no coeficiente de performance (COP) de Carnot para estabelecer um referencial teórico; posteriormente, desenvolveu-se um modelo de ciclo real incorporando dados operacionais dos compressores selecionados, avaliando diferentes condições de evaporação e condensação conforme os catálogos técnicos; finalmente, implementou-se um procedimento de otimização que permite a iteração simultânea das temperaturas de evaporação e condensação, identificando o ponto ótimo de operação que minimiza o consumo energético do sistema.

Este trabalho apresenta, portanto, uma abordagem sistemática para o projeto de sistemas de refrigeração, integrando análise termodinâmica, dados de fabricantes e técnicas de otimização para a tomada de decisão fundamentada na seleção de compressores.