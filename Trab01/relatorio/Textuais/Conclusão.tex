\chapter{Conclusão}

Este trabalho apresentou uma metodologia sistemática para o dimensionamento e seleção de compressores para sistemas de refrigeração aplicados ao armazenamento de pescado. A abordagem desenvolvida, partindo da análise do ciclo ideal de Carnot até a modelagem de ciclos reais com dados de fabricantes, demonstrou-se adequada para avaliar o desempenho termodinâmico de diferentes compressores candidatos.

A implementação computacional permitiu automatizar os cálculos e comparar múltiplas condições operacionais de forma eficiente, facilitando a identificação de configurações que equilibrem capacidade frigorífica e eficiência energética. A análise de geração de entropia complementou o estudo ao quantificar as irreversibilidades em cada componente, revelando que o evaporador é responsável pela maior parcela das perdas termodinâmicas do sistema, o que orienta futuras melhorias de projeto focadas na otimização da transferência de calor neste componente.

A metodologia pode ser facilmente adaptada para outras aplicações de refrigeração com diferentes produtos, temperaturas e capacidades, constituindo uma ferramenta útil para projetos de sistemas de refrigeração que considere tanto aspectos de desempenho quanto eficiência energética.